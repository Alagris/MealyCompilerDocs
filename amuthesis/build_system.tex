To guarantee the utility of Solomonoff, it has to provide ability of interactive evaluation in order to enable efficient and convenient work with the compiler. It can be also vastly useful while learning. The mentioned feature is implemented with a REPL. The biggest challenge here is adjusting the entire stack of compilation processes to work interactively with a sensible manner. The other condition of usefulness is a fast build automation. To make our system meet the
need it has to support both parallel compilation and caching of previously compiled units. Also the ability to resolve dependencies may be be important in some more complex projects. It this case the main challenge is to find satisfying solution of ordering in a parallel compilation. This problem can be broken down to a topological sort in a direct acyclic graph.
To make the use of Solomonoff even more practical it provides an export option for transducers in a way that allows to include them into a C code during a recompilation phase. 
