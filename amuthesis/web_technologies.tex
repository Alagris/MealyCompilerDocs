
Creation of user-friendly interface was an important part of the project.
The greatest challenge lied in finding the most intuitive way of presenting a complicated and highly advanced system. The main component was the language of regular expressions itself. User should be able to edit its code with ease. The second key feature was the ability to execute the code. 
In many Turing-complete languages, every expression can be evaluated into some value, which could the be printed back to the user. Here the problem was not so trivial. The regular expressions could in principle be evaluated down to formal languages. For example 
\begin{lstlisting}
'a' ('b' | 'c' | 'ef' ) 'd'
\end{lstlisting}
would return a language consisting of strings
\begin{lstlisting}
'abd', 'acd', 'aefd'
\end{lstlisting}
The issue wish such approach is that not all languages are finite. The expression
\begin{lstlisting}
'a'*
\end{lstlisting}
would be evaluated as infinite set
\begin{lstlisting}
'', 'a', 'aa', 'aaa', ...
\end{lstlisting}
Some regexes, might be finite but of exponential size. For instance
\begin{lstlisting}
('0' | '1') ('0' | '1') ('0' | '1') ('0' | '1')
\end{lstlisting}
yields set of all bit-strings of length 4. Presenting user with the result in form of formal languages would be often impractical or impossible. 

As a result, our REPL does not evaluate expressions. The results of compilation are not printed in any form. Instead the interface is meant to be silent when compilation is successful. Only errors are printed. 

There are many different approaches to implement user interface for REPL.
One of them would be having a single editor window with all the code in it and the REPL output printed on the margins next to each respective line. This provides a very immersive user experience for Turing-complete languages. For regular expressions i'ts not as spectacular. Instead we decided to use two windows - one for code editor and the other for REPL console. All interaction with regular expressions is performed via special commands built into the console. Those commands could not be used inside the code editor, as they are not part of the language itself. In order to evaluate a transducer user would type the following line into REPL input
\begin{lstlisting}
:eval NAME 'input string'
\end{lstlisting}
Sometimes, user might want to see all the strings that belong to a given language. While there might be infinitely many of them, it's possible to ask user how large sample to generate. To achieve this the following line can be used
\begin{lstlisting}
:rand_sample NAME of_size NUMBER
\end{lstlisting}
Automata can also be interpreted as directed graphs. This property makes was used to further enhance user interface. Automaton's graph will be shown after typing this command
\begin{lstlisting}
:vis NAME
\end{lstlisting}
Those and many other functionalities have been implemented in the browser-based version of REPL. 

The implementation is not trivial. One of the ways to achieve such results would be by implementing a parser that could halt mid-parsing. For example user could first type
\begin{lstlisting}
x = ('x' |
\end{lstlisting}
and hit return button. The parser should notice that the expression is not finished and it has to wait for the next line of input.  Then as the user types the next line
\begin{lstlisting}
'y' )
\end{lstlisting}
a full and valid expression could be recognised and parser could return.
This approach is used by some programming languages. It's difficult to implement and requires the grammar to be appropriately structured. We later abandoned this idea due to the problematic nature of Solomonoff's grammar. In particular, it does not use semicolons to separate statements. For example
\begin{lstlisting}
x = 'a' 
y = 'b'
\end{lstlisting}
could be written in a single line
\begin{lstlisting}
x = 'a' y = 'b'
\end{lstlisting}
The equality sign determines start of new statement. By its very nature, parsing this, requires a lookahead of one token into the future. When the input is read in fragments, line by line, such a lookahead is not possible to obtain.
User could first type
\begin{lstlisting}
x = 'a' y
\end{lstlisting}
which would be recognized by parser as concatenation of string \texttt{'a'} with variable \texttt{y}. If the user then types
\begin{lstlisting}
 = 'b'
\end{lstlisting}
in the upcoming line, then the previous results of parsing would have to be discarded and the entire input reparsed again. Hence we decided to simplify the REPL and assume that every line of input fully defines the entirety of expression. As a result it's not possible to split input into multiple lines when using console. This is not a serious limitation, because multiline expressions could still be written in the editor window instead of console. 

The division of user interface into editor and console has one more advantage. It closely mimics the layout of command-line interface, where the typical workflow is to edit source code in local files using any text editor of user's choice and the REPL is kept open all the time alongside the editor. Many existing modes for Emacs follow similar convention.

The REPL is implemented on the server-side as a REST API endpoint. 
\begin{lstlisting}
@PostMapping("/repl")
public ReplResponse repl(HttpSession httpSession, 
    @RequestBody String line)
\end{lstlisting}
Every user has their own instance of REPL
\begin{lstlisting}
Repl repl = (Repl) httpSession.getAttribute("repl");
\end{lstlisting}
which holds a reference to the compiler and a set of built-in commands
\begin{lstlisting}
 public static class Repl {
    private static class CmdMeta<Result> {
        final ReplCommand<Result> cmd;
        final String help;
        final String template;
        
        private CmdMeta(ReplCommand<Result> cmd, 
               String help, String template) {
            this.cmd = cmd;
            this.help = help;
            this.template = template;
        }
    }
	
    HashMap<String, Repl.CmdMeta<String>> commands;
    OptimisedHashLexTransducer compiler;
}
\end{lstlisting}
whenever user types some command on the REPL console
\begin{lstlisting}
:cmd arg1 arg2 arg3
\end{lstlisting}
it gets parsed as
\begin{lstlisting}
String firstWord = "cmd";
String remaining = "arg1 arg2 arg3";
\end{lstlisting}
and then the appropriate command implementation is looked up in the map
\begin{lstlisting}
final Repl.CmdMeta<String> cmd = commands.get(firstWord);
return cmd.cmd.run(httpSession, compiler, log, debug, remaining);
\end{lstlisting}
The rest controller contains implementations of many such commands
\begin{lstlisting}
public static final ReplCommand<String> REPL_LIST = ...
public static final ReplCommand<String> REPL_EVAL = ...
public static final ReplCommand<String> REPL_RUN = ...
public static final ReplCommand<String> REPL_EXPORT = ..
public static final ReplCommand<String> REPL_IS_DETERMINISTIC = ...
public static final ReplCommand<String> REPL_LIST_PIPES = ...
public static final ReplCommand<String> REPL_EQUAL = ...
public static final ReplCommand<String> REPL_RAND_SAMPLE = ...
public static final ReplCommand<String> REPL_CLEAR = ...
public static final ReplCommand<String> REPL_UNSET = ...
public static final ReplCommand<String> REPL_RESET = ...
public static final ReplCommand<String> REPL_LOAD = ...
public static final ReplCommand<String> REPL_VIS = ...
\end{lstlisting}
All of those definitions above are lambda expressions that use library functions
of the compiler. The parameters taken by those lambda expressions are as follows
\begin{lstlisting}
public interface ReplCommand<Result> {
    Result run(
        HttpSession httpSession, 
        OptimisedHashLexTransducer compiler, 
        Consumer<String> log, 
        Consumer<String> debug, 
        String args) throws Exception;
}
\end{lstlisting}
As an example, consider the command 
\begin{lstlisting}
:eval f 'abc'
\end{lstlisting}
which evaluates transducer \texttt{f} for input string \texttt{'abc'}. 
On the frontend JavaScript will perform REST query
\begin{lstlisting}
const response = await fetch('repl', {
	method: 'POST',
	body: ":eval f 'abc'"
})
\end{lstlisting}
which will be received by server
\begin{lstlisting}
@PostMapping("/repl")
public ReplResponse repl(HttpSession httpSession, 
        @RequestBody String line) {
    Repl repl = (Repl) httpSession.getAttribute("repl");
    final String result = repl.run(
        httpSession, 
        line, // ":eval f 'abc'"
        s -> out.append(s).append('\n'), // console output
        s -> { } // debug logs are not displayed
    );
    ...
}
\end{lstlisting}
and the \texttt{repl.run} method will query the appropriate implementation to call for
the \texttt{:eval} command.
\begin{lstlisting}	
final String firstWord = "eval";
final String remaining = "f 'abc'";
final Repl.CmdMeta<String> cmd = commands.get(firstWord);
return cmd.cmd.run(httpSession, compiler, log, debug, remaining);
\end{lstlisting}
This in turn will trigger the following lambda function
\begin{lstlisting}
ReplCommand<String> REPL_EVAL = 
        (httpSession, compiler, logs, debug, args) -> {
    String[] parts = args.split("\\s+", 2); // f 'abc'
    String name = parts[0]; // f
    String input = parts[1]; // 'abc'
    G transducer = compiler.getTransducer(name);
    String output = compiler.specs.evaluate(transducer, input);
    return output == null ? "No match!" : output;
};
\end{lstlisting}
The output is sent back to JavaScript in form of JSON
\begin{lstlisting}
const replResult = JSON.parse(await response.text())
\end{lstlisting}
Remaining commands are implemented in a similar way.

Several of the functions may require access to HTTP session. In particular it's worth analysing the \texttt{:load} command. It's purpose is to emulate the process of loading source code from file. Whenever user types some code in the editor, it needs to be transported to server, then parsed and compiled. All the defined transducers results need to be saved for later use. The simplest way of achieving this, would be by extending the REST API as
\begin{lstlisting}
class ReplInput{
	String command;
	String editorContent;
}
public ReplResponse repl(
    HttpSession httpSession, 
    @RequestBody ReplInput)
\end{lstlisting}
and query it using
\begin{lstlisting}
const response = await fetch('repl', {
    method: 'POST',
    body: {
        command: replCommand,
        editorContent: editor.getValue()
    }
})
\end{lstlisting}
The downside of such solution is that the editor content could become large and sending it would require more internet bandwidth and time. Often user only wants to execute simple short REPL commands that do not require sending the entire code. Sometimes the code might be required but resending it might be omitted as long as user has not modified it. Hence the process of uploading code to the server has been delegated to a separate REST call.
\begin{lstlisting}
const response = await fetch('upload_code', {
	method: 'POST',
	body: code
})
\end{lstlisting}
The code is then stored in HTTP session, so the REST endpoint has a very simple implementation
\begin{lstlisting}
@PostMapping("/upload_code")
public void uploadCode(HttpSession httpSession, 
        @RequestBody String text) {
    httpSession.setAttribute("code", text);
}
\end{lstlisting}


Aside from the editor and REPL there is one more window on the webpage. It is dedicated for tutorial and short documentation.  While it does not enhance the functionality of the website per se, it plays an important role. The Solomonoff compiler is a very niche and specialised tool. There are no similar tools and any user coming to the website is not expected to be familiar with its usage. The primary purpose of the website is not to be a replacement for user's IDE and terminal. Instead it serves as an all-in-one introductory tutorial, interactive playground and a marketing campaign. We want to make the learning materials easily accessible and abundant. Building a strong community is the back-bone of every open-source project. 












